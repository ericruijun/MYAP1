
\documentclass[11pt,a4paper]{article}

\usepackage[utf8]{inputenc} 
\usepackage[T1]{fontenc} 
\usepackage{lmodern}
\usepackage{tcolorbox}

\usepackage[german]{babel}


\setlength{\parindent}{0pt}
\setlength{\parskip}{1ex plus 0.5ex minus 0.5ex}

\usepackage{amsmath} 


\usepackage{graphicx} 

\usepackage[section]{placeins}
\usepackage{booktabs}


\usepackage{hyperref}
\hypersetup{
	colorlinks,
	citecolor=red,
	filecolor=black,
	linkcolor=black,
	urlcolor=black}
\graphicspath{}

\begin{document}
	

{
	\centering 
	\large 
	Physiklabor für Anfänger*innen \\
	Ferienpraktikum im Sommersemester 2018 \\[4mm]
	\textbf{\LARGE 
		Versuch 8: Viskosität aus dem Durchströmen einer Kapillare
	} \\[3mm]
	(durchgeführt am 26.09.2018 bei ) \\
	Ye Joon Kim, Marouan Zouari\\
	\today \\[10mm]
}
\tableofcontents

\section{Aufbau}


\section{Durchführung}

\section{Auswertung und Fehleranalyse}
Zur Bestimmung der Viskosität wurde die folgende Formel benutzt:
\begin{equation}
	I_V = \frac{V}{t} = \frac{\pi R^4 \Delta p}{8 \eta l}
\end{equation}

\section{Diskussion der Ergebnisse}

\section{Literatur}
	

	
	
	
	
	
	
	
	
	
	
	
	
	
	
	
\end{document}