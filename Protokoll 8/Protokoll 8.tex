
\documentclass[11pt,a4paper]{article}

\usepackage[utf8]{inputenc} 
\usepackage[T1]{fontenc} 
\usepackage{lmodern}
\usepackage{tcolorbox}

\usepackage[german]{babel}


\setlength{\parindent}{0pt}
\setlength{\parskip}{1ex plus 0.5ex minus 0.5ex}

\usepackage{amsmath} 


\usepackage{graphicx} 

\usepackage[section]{placeins}
\usepackage{booktabs}


\usepackage{hyperref}
\hypersetup{
	colorlinks,
	citecolor=red,
	filecolor=black,
	linkcolor=black,
	urlcolor=black}
\graphicspath{}


\begin{document}


{
	\centering 
	\large 
	Physiklabor für Anfänger*innen \\
	Ferienpraktikum im Sommersemester 2018 \\[4mm]
	\textbf{\LARGE 
		Versuch 6: Bestimmung des Elastizitätsmoduls aus der Biegung
	} \\[3mm]
	(durchgeführt am 24.09.2018 bei Julia Müller) \\
	Ye Joon Kim, Marouan Zouari\\
	\today \\[10mm]
}
\tableofcontents

\section{Einleitung}
Wird ein Körper von einer äußeren Kraft belastet, ändert sich seine Länge. Die Längenänderung ist durch eine Proportionalität, $E$, mit der wirkende Kraft pro Fläche gekoppelt. Diese Proportionalität ist als der Elastizitätsmodul bezeichnet. 
\begin{equation}
\frac{\Delta l}{l} = \frac{1}{E}\frac{F}{A} = \frac{1}{E}\sigma
\end{equation}
Wobei $\sigma = \frac{F}{A}$ die Zugspannung ist. Diese Gleichung ist auch als das Hook'sche Gesetz bekannt.

Wird ein Horizontaler Stab von einer Kraft nach oben belastet, biegt sich der Stab. Die obere und untere Hälfte werden jeweils gestaucht und gedehnt und von den resultierenden inneren entgegengesetzten Spannungen entsteht ein Drehmoment. Da dieses Drehmoment von der Durchbiegung abhängt, kann die Elastizitätsmodul von der Durchbiegung, oder Biegungpfeil bestimmt werden, nämlich mit der folgenden Formel:
\begin{equation}
s = \frac{1}{E}\frac{l^3}{4h^3b}F
\end{equation}
Wobei:
\begin{itemize}
	\item $s$ der Biegungspfeil
	\item $l$ die Länge 
	\item $h$ die Höhe
	\item $b$ die Breite des Stabes
	\item $F$ die auf den Stab wirkende Kraft sind.
\end{itemize}
\section{Aufbau}

\section{Versuchsdurchführung}
Zuerst wurden die Dimensionen der Stäbe (Breite und Höhe) gemessen. Jede Messung wurde fünfmal wiederholt. Danach wurden die einzelnen in dem Versuch verwendeten Gewichtsstücke gewogen und deren Massen aufgenommen. 

Für jede Messreihe wurde der Stab zuerst auf die zwei Schneiden gelegt. Es wurde mit dem Maßband sichergestellt, dass die Schneiden gleich entfernt von dem Mittelpunkt waren. Die Spitze des Messgeräts wurde auf die Halterung getan und geeicht. Es wurde dann eine relative große Masse auf die Halterung aufgehängt (In diesem Fall 6 Große Gewichtsstücke). Der Abstand zwischen die Schneiden wurde dann geändert, sodass die angezeigte Durchbiegung zwischen 2 und 3mm lag. Danach wurden die Gewichtsstücke entfernt. 

Für die einzelnen Messungen wurde auf die Halterung die Gewichtsstücke aufgehängt. Für jede Gewichtsänderung wurde der auf dem Messgerät angezeigte Wert aufgenommen. Die Belastung wurde von kleineren zu größeren Werten variiert und wieder in umgekehrter Reihenfolge, um zwei Messwerte für jede Gewicht aufnehmen zu können. 

Dieser Prozesse wurde für unterschiedliche Materialien, Ausrichtungen des Stabs und  Längen zwischen den Schneiden wiederholt. 






\section{Auswertung und Fehleranalyse}
Für jede Messreihe wurden die jeweiligen Mittelwerte in einer Graph aufgetragen. 

\begin{figure}[h]
	\centering
	\includegraphics[width=\linewidth]{Abb2}
	\caption{ayy lmao}
\end{figure}

\begin{figure}[h]
	\centering
	\includegraphics[width=\linewidth]{Abb3}
	\caption{ayy lmao}
\end{figure}

\begin{figure}[h]
	\centering
	\includegraphics[width=\linewidth]{Abb4}
	\caption{ayy lmao}
\end{figure}
\FloatBarrier
Mit einem Excel-Dokument wurde die lineare Regressionen und deren Unsicherheit berechnet. Die lineare Zusammenhänge lassen sich in der Form: 
$$ s = a + bm $$ schreiben. Die einzelne Werte für $a$ und $b$ für jede Messreihe sind in Tabelle 1 bis 3 zu sehen (Achtung, dass in diesem Kontext $b$ nicht die Breite des Stabes ist). 

\begin{table} [h]
	\begin{tabular*}{0.99\textwidth}{@{\extracolsep{\fill}}c|cccccc}
		\toprule
		Material & $a$ & $u_a$ & $b$ & $u_b$  \\
		 & mm & mm & mm kg$^{-1}$ & mmkg$^{-1}$ & \\
		\bottomrule
		Stahl & -0,02 & 0,07 & 2,0 & 0,2 \\
		Aluminium & -0,006 & 0,006 & 2,352 & 0,008 \\
		Messing & 0,1 & 0,1 & 2,6 & 0,2 \\
		\bottomrule
	\end{tabular*}
	\caption{Werte für $a$ und $b$ und deren Unsicherheit für verschiedene Materialien}
\end{table}

\begin{table} [h]
	\begin{tabular*}{0.99\textwidth}{@{\extracolsep{\fill}}c|cccccc}
		\toprule
		Ausrichtung & $a$ & $u_a$ & $b$ & $u_b$  \\
		& mm & mm & mm kg$^{-1}$ & mmkg$^{-1}$ & \\
		\bottomrule
		Horizontal & -0,006 & 0,006 & 2,352 & 0,008 \\
		Vertikal & -0,003 & 0,005 & 1,048 & 0,006 \\
		
		\bottomrule
	\end{tabular*}
	\caption{Werte für $a$ und $b$ und deren Unsicherheiten für unterschiedliche Ausrichtungen des Stabprofils}
\end{table}

\begin{table} [h]
	\begin{tabular*}{0.99\textwidth}{@{\extracolsep{\fill}}c|cccc}
		\toprule
		Abstand & $a$ & $u_a$ & $b$ & $u_b$  \\
		& mm & mm & mm kg$^{-1}$ & mmkg $^{-1}$  \\
		\bottomrule
		Horizontal & -0,006 & 0,006 & 2,352 & 0,008 \\
		Vertikal & -0,003 & 0,005 & 1,048 & 0,006 \\
		
		\bottomrule
	\end{tabular*}
	\caption{Werte für $a$ und $b$ und deren Unsicherheiten für unterschiedliche Abstände zwischen den Schneiden}
\end{table}
\FloatBarrier
Mit Formel (2) und $F = mg$ kann man leicht sehen, dass die Steigung $b$ dem Wert $\frac{1}{E}\frac{gl^3}{4h^3b}$ Entspricht. Dadurch lässt sich der Wert von $E$ berechnen:
$$ E = \frac{gl^3}{4h^3bS}$$
In dieser Formel ist $S$ die Steigung, um Verwechslung mit dem Buchstabe für Breite zu vermeiden. Die Berechnete Werte für $E$ und deren Unsicherheiten für jede Messreihe sind:



\section{Diskussion der Ergebnisse}

\section{Literatur}
	
	
	
	
	
	
	
	
\end{document}