
\documentclass[11pt,a4paper]{article}

\usepackage[utf8]{inputenc} 
\usepackage[T1]{fontenc} 
\usepackage{lmodern}
\usepackage{tcolorbox}
\usepackage[german]{babel}


\setlength{\parindent}{0pt}
\setlength{\parskip}{1ex plus 0.5ex minus 0.5ex}

\usepackage{amsmath} 


\usepackage{graphicx} 

\usepackage[section]{placeins}
\usepackage{booktabs}


\usepackage{hyperref}
\hypersetup{
	colorlinks,
	citecolor=red,
	filecolor=black,
	linkcolor=black,
	urlcolor=black}
\graphicspath{}

\begin{document}
	

{
	\centering 
	\large 
	Physiklabor für Anfänger*innen \\
	Ferienpraktikum im Sommersemester 2018 \\[4mm]
	\textbf{\LARGE 
		Versuch 70: Linsen und Linsensysteme
	} \\[3mm]
	(durchgeführt am 28.09.2018 bei Daniel Bartle) \\
	Ye Joon Kim, Marouan Zouari\\
	\today \\[10mm]
}
\tableofcontents

\section{Einleitung}
Mit einer Linse kann man durch die Brechung Licht ablenken. Jede (richtig hergestellte) Linse besitzen zwei Brennpunkte, wo alle parallelen und zur Linse senkrecht einfallenden Lichtstrahlen sich sammeln. Der Abstand von dem Mittelpunkt der Linse zu einem Brennpunkt heißt die Brennweite. Für dicke Linsen und Linsensysteme muss einen weiteren Begriff eingeführt werden. Die doppelte Brechung lassen sich durch Hauptebenen beschreiben. Zwischen den Hauptebenen können die Lichtstrahlen als parallel verlaufend gedacht. Diese Begriffe vereinfachen Berechnungen mit Lichtstrahlen. 

Es existieren mehrere Verfahren, um die Brennweite von Linsen und Linsensysteme zu bestimmen. Bei einer Einzellinse kann die Brennweite $f$ so ausgedruckt werden: 

\begin{equation}
\frac{1}{f} = \frac{1}{g} + \frac{1}{b}
\end{equation}
Wobei $g$ und $b$ die Gegenstandsweite bzw. Bildweite, die Abstände von jeweils dem Gegenstand und Bild zur Linsenmitte, sind.
Für ein Linsensystem, das aus zwei Linsen besteht, gilt eine andere Formel, nämlich:
\begin{equation}
\frac{1}{f} = \frac{1}{f_1}+\frac{1}{f_2}-\frac{d}{f_1f_2}
\end{equation}
Wobei $f_1$ und $f_2$ die Brennweiten der einzelnen Linsen und $d$ der Abstand dazwischen sind. 

Aber wenn die Hauptebenen nicht bekannt sind, können $g$ und $b$ nicht direkt bestimmt werden. In diesem Fall hilft das Bessel-Verfahren, wobei die Brennweite mit:
\begin{equation}
f = \frac{s^2-e^2}{4s}
\end{equation}
bestimmt werden kann, wobei $s$ der Abstand zwischen dem Gegenstand und Bild, und $e$ die Differenz der Positionen, wo Abbildungen möglich sind. 

Mit einem anderen Verfahren, lassen sich die Brennweite und die Hauptebenen gleichzeitig bestimmen, dieses Verfahren heißt das Abbe-Verfahren. Für das Abbe-Verfahren wird einen Punkt ausgewählt und im Referenz zu diesem Punkt die scheinbare Gegenstandsweite $g'$ und die scheinbare Bildweite $b'$ gemessen. Hier wird außerdem der Abbildungsmaßstab verwendet: 
$$ \beta = \frac{B}{G} = \frac{b}{g}$$
Wobei $B$ und $G$ jeweils de Größen des Bildes und Gegenstands sind. 
Mit direkten Messungen von $B$, $G$, $g'$ und $b'$ kann mit den Gleichungen:
\begin{equation}
\begin{array}{l}
	g' = (1+\frac{1}{\beta})\cdot f_1 + h_1 \\
	b' = (1+\beta)\cdot f_2 + h_2
\end{array}
\end{equation}
die $f$ und $h$ bestimmt werden, da $f$ die Steigungen, und $h$ den Achsenabschnitte entsprechen. 



\section{Aufbau und Durchführung}



\section{Auswertung und Fehleranalyse}
\subsection{1. Versuchsteil: Abbildungen mit Einzellinsen und Linsensystemen}
Die Werte für $\frac{1}{b}$ wurden gegen $\frac{1}{g}$ aufgetragen. Für jede Messreihe wurde auch lineare Regressionen durchgeführt, aber nur, um die Linearität der Zusammenhang zu veranschaulichen (Siehe Abbildung 1). Die theoretische Verläufe wurden auch mit Formel (2) und (1) berechnet (Siehe Abbildung 2). 
\begin{figure}
	\centering
	\includegraphics[width=\linewidth]{Abb2}
	\caption{$\frac{1}{b}$ gegen $\frac{1}{g}$.}
\end{figure} 

\begin{figure}
	\centering
	\includegraphics[scale=0.5]{Abb3}
	\caption{ Die aus der Abbildungsgleichung Erwarteten Linearen Verläufe}
\end{figure} 

\begin{tcolorbox}[colback=white]
\subsubsection{Rechenweg}
Zuerst wurde die Unsicherheiten von $g$ und $b$ mit der vereinfachten gauß'schen Fehlerfortpflanzung für Summe berechnet. Zum Beispiel:
$$ \Delta g = \sqrt{(\Delta x_\textrm{Dia})^2 + (\Delta x_\textrm{Groß})^2}$$
, da $g = x_\textrm{Groß} - x_\textrm{Dia}$, wobei $x_\textrm{Dia}$ die Position des Dias und $x_\textrm{Groß}$ die Position der Linse, wo eine vergrößernde Abbildung möglich ist. 

Die Unsicherheiten der einzelnen Punkten auf der Graph wurde mit der vereinfachten gaußschen Fehlerfortpflanzung für Produkte und Quotienten bestimmt.
$$ \Delta\left(\frac{1}{g}\right) = \frac{\Delta g}{g} \cdot \frac{1}{g}$$
ebenfalls für die $\frac{1}{b}$ Werte. 
	
Für die theoretischen Verläufe wurde Gleichung (1) nach $\frac{1}{b}$ umgeformt und die gesamten Brennweiten mit Gleichung (2) Berechnet. 
	
	
\end{tcolorbox}

\subsection{2. Versuchsteil: Das Bessel-Verfahren}
Gemäß Gleichung (3) wurden die Werte für $s$ und $e$ und deren Unsicherheiten berechnet (Siehe Anhang 1). Die Unsicherheiten wurde wie in dem ersten Versuchsteil mit der gauß'sche Fehlerfortpflanzung für Summe berechnet. Danach wurden die einzelne Werte für $f$ für jede Messreihe berechnet (Siehe Anhang 2). 

Die Unsicherheiten der $f$ Werte wurden mit der gauß'schen Fehlerfortpflanzung berechnet. Mit 
$$ f = \frac{s^2-e^2}{4s}$$
sind:
$$ \frac{\partial f}{\partial{s}} = \frac{s^2+e^2}{4s^2}$$
$$\frac{\partial f}{\partial{e}} = -\frac{e}{2s}$$
Die Unsicherheit von $f$ ist deshalb:
$$\Delta f = \sqrt{(\frac{\partial f}{\partial{s}}\Delta s)^2 + (\frac{\partial f}{\partial{e}} \Delta e)^2}$$

Die Mittelwerte der $f$ für jede Linse/Linsensystem und deren Standardunsicherheiten sind dann:


 



\begin{table} [h]
	\centering
	\begin{tabular*}{0.50\textwidth}{@{\extracolsep{\fill}}c|cccccc}
		\toprule
		Brennweite & $f$ & $u_f$   \\
		mm & cm & cm \\
		\bottomrule
		80 & 7,93 & 0,14 \\
		80 \& 150 & 5,63 & 0,07 \\
		80 \& -200 & 11,59 & 0,05 \\
		\bottomrule
	\end{tabular*}
	\caption{Die Berechneten $f$ für die Linsen und Linsensysteme}
\end{table}

\subsection{3. Versuchsteil: Das Abbe-Verfahren}
Zuerst wurden die Werte für $\beta$ ausgerechnet (Siehe Tabelle 1).


\begin{table}[h]
	\centering
	\begin{tabular*}{0.50\textwidth}{@{\extracolsep{\fill}}cc|ccccc}
		\toprule
		$g'$ & $b'$ & $\beta$ & $\Delta \beta$   \\
		cm & cm& &\\
		18,5&47,9&3,33&0,05\\
		20,8&28,6&1,71&0,05\\
		19,7&25,0&2,19&0,05\\
		17,8&47,2&3,71&0,05\\
		15,3&74,7&4,71&0,14\\
		16,1&90,3&6,71&0,14\\
		16,0&101,8&7,86&0,14\\
		17,3&54,6&3,79&0,07\\
		18,3&41,7&2,71&0,07\\
		18,4&39,0&2,57&0,14\\
		\bottomrule
	\end{tabular*}
	\caption{Die gemessenen Werte für $g'$, $b'$ und $\beta$ für ein Linsensystem mit $f_1 = 80mm$ und $f_2 = -200mm$}
\end{table}



\begin{table}[h]
	\centering
	\begin{tabular*}{0.50\textwidth}{@{\extracolsep{\fill}}cc|ccccc}
		\toprule
		$g'$ & $b'$ & $\beta$ & $\Delta \beta$   \\
		cm & cm& &\\
		13,5&58,6&2,86&0,14\\
		13,9&51,5&3,21&0,07\\
		14,6&44,3&2,64&0,07\\
		21,4&25,3&1,00&0,07\\
		12,4&93,6&6,86&0,14\\
		12,4&84,5&6,14&0,14\\
		12,0&108,6&8,14&0,14\\
		12,9&76,0&8,00&0,14\\
		12,8&68,6&4,64&0,07\\
		14,3&48,6&3,00&0,07\\
		\bottomrule
	\end{tabular*}
	\caption{Die gemessenen Werte für $g'$, $b'$ und $\beta$ für ein Linsensystem mit $f_1 = -200mm$ und $f_2 = 80mm$}
\end{table}

\FloatBarrier
Danach wurden $g'$ und $b'$ jeweils gegen $(1+\frac{1}{\beta})$ und $(1+\beta)$ aufgetragen (Siehe Anhang 3, 4). 
Dadurch lassen sich die Werte für $f_1$, $f_2$, $h_1$ und $h_2$ für beide Linsensysteme bestimmen. 
Das System mit zuerst einer Linse mit $f=80mm$ und danach einer mit $f=-200mm$ hat die Folgenden Werte:
$$f_1 = 11\textrm{cm    } f_2 = 12\textrm{cm}$$
$$ h_1 = 3\textrm{cm    } h_2 = 8 \textrm{cm}$$

Und für das System mit umgekehrter Reihenfolge der Linsen:
$$f_1 = 3\textrm{cm    } f_2 = 10 \textrm{cm}$$
$$ h_1 = 10\textrm{cm    } h_2 = 18 \textrm{cm}$$


\subsection{4. Versuchsteil: Autokollimationsverfahren und Dispersion}
Der Abstand zwischen der Linse und dem Objekt, wobei eine Scharfe Abbildung möglich war, wurde für verschiedene Linsen und Linsensysteme und danach mit einer Linse mit verschiedenen Lichtfarben berechnet. Dieser Abstand entspricht dann der Brennweite. 

\begin{table}[h]
	\centering
	\begin{tabular*}{0.75\textwidth}{@{\extracolsep{\fill}}c|cc}
		\toprule
		Theoretische Brennweite & $f$ & $\Delta f$ \\
		der einzelnen Linsen &&\\
		mm & cm & cm \\
		80  & 7,6 & 0,3 \\
		150 & 24,6 & 0,3 \\
		80 \& 150 & 4,6 & 0,3 \\
		80 \& -200 & 12,2 & 0,3 \\
		-200 \& 80 & 8,0 & 0,3\\
		\bottomrule
	\end{tabular*}
	\caption{Die mit dem Autokollimationsverfahren bestimmten $f$ für verschiedene Linsen und Linsensysteme.}
\end{table}

\begin{table}[h]
	\centering
	\begin{tabular*}{0.75\textwidth}{@{\extracolsep{\fill}}c|cc}
		\toprule
		Lichtfarbe & $f$ & $\Delta f$ \\
		 & cm & cm \\
		Weiß  & 33,6 & 0,3 \\
		Rot & 33,5 & 0,3 \\
		Blau & 32,8 & 0,3 \\
		\bottomrule
	\end{tabular*}
	\caption{Die mit dem Autokollimationsverfahren bestimmten $f$ für verschiedene Lichtfarben}
\end{table}

Zur Bestimmung der Unsicherheiten wurde die vereinfachte Fehlerfortpflanzung für Summe verwendet. Deswegen ist $\Delta f$:
$$\sqrt{2\cdot(0,2 \textrm{cm})^2} \approx 0,28$$

\section{Diskussion der Ergebnisse}

\section{Anhang}


\begin{table}[h]
	\centering
	\begin{tabular*}{0.50\textwidth}{@{\extracolsep{\fill}}cc|ccccc}
		\toprule
		Brennweite & Messreihe & $s$ & $e$   \\
		mm &  & cm & cm  \\
		\bottomrule
		80 & 1 & 57,7 & 39,3 \\
		& 2 & 72,4 & 54,4 \\
		& 3 & 88,7 & 70,8 \\
		& 4 & 106,4 & 89,2 \\
		& 5 & 67,7 & 48,9 \\
		80 \& 150 & 1 & 67,7 & 55,3 \\
		& 2 & 83,9 & 71,8 \\
		& 3 & 34,1 & 19,5 \\
		& 4 & 24,0 & 5,9 \\
		& 5 & 39,0 & 25,6 \\
		80 \& -200 & 1 & 95,7 & 68,6 \\
		& 2 & 74,8 & 46,2 \\
		& 3 & 120,7 & 94,9 \\
		& 4 & 59,6 & 28,2 \\
		& 5 & 51,3 & 15,5 \\
		\bottomrule
	\end{tabular*}
	\caption{Die Werte für $s$ und $e$ für alle Messreihen}
\end{table}


\begin{table}[h]
	\centering
	\begin{tabular*}{0.50\textwidth}{@{\extracolsep{\fill}}cc|ccccc}
		\toprule
		Brennweite & Messreihe & $s$ & $e$   \\
		mm &  & cm & cm  \\
		\bottomrule
		80 & 1 & 7,73 & 0,14 \\
		& 2 & 7,88 & 0,15 \\
		& 3 & 8,05 & 0,16 \\
		& 4 & 7,90 & 0,17 \\
		& 5 & 8,09 & 0,15 \\
		80 \& 150 & 1 & 5,63 & 0,17 \\
		& 2 & 5,61 & 0,17 \\
		& 3 & 5,73 & 0,12 \\
		& 4 & 5,64 & 0,08 \\
		& 5 & 5,55 & 0,14 \\
		80 \& -200 & 1 & 11,63 &0,15 \\
		& 2 & 11,56 & 0,13 \\
		& 3 & 11,52 & 0,16 \\
		& 4 & 11,56 & 0,11 \\
		& 5 & 51,3 & 0,09 \\
		\bottomrule
	\end{tabular*}
	\caption{Berechnete Werte für $f$ und deren Unsicherheiten}
\end{table}

\begin{figure}[h]
	\centering
	\includegraphics[width=\linewidth]{Abb4}
	\caption{Graph von $g'$ gegen $1+\frac{1}{\beta}$ für beide Linsensysteme}
\end{figure}

\begin{figure}[h]
	\centering
	\includegraphics[width=\linewidth]{Abb5}
	\caption{Graph von $b'$ gegen $1+\beta$ für beide Linsensysteme}
\end{figure}

\end{document}